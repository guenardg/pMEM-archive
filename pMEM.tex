% Options for packages loaded elsewhere
\PassOptionsToPackage{unicode}{hyperref}
\PassOptionsToPackage{hyphens}{url}
%
\documentclass[
]{article}
\usepackage{amsmath,amssymb}
\usepackage{iftex}
\ifPDFTeX
  \usepackage[T1]{fontenc}
  \usepackage[utf8]{inputenc}
  \usepackage{textcomp} % provide euro and other symbols
\else % if luatex or xetex
  \usepackage{unicode-math} % this also loads fontspec
  \defaultfontfeatures{Scale=MatchLowercase}
  \defaultfontfeatures[\rmfamily]{Ligatures=TeX,Scale=1}
\fi
\usepackage{lmodern}
\ifPDFTeX\else
  % xetex/luatex font selection
\fi
% Use upquote if available, for straight quotes in verbatim environments
\IfFileExists{upquote.sty}{\usepackage{upquote}}{}
\IfFileExists{microtype.sty}{% use microtype if available
  \usepackage[]{microtype}
  \UseMicrotypeSet[protrusion]{basicmath} % disable protrusion for tt fonts
}{}
\makeatletter
\@ifundefined{KOMAClassName}{% if non-KOMA class
  \IfFileExists{parskip.sty}{%
    \usepackage{parskip}
  }{% else
    \setlength{\parindent}{0pt}
    \setlength{\parskip}{6pt plus 2pt minus 1pt}}
}{% if KOMA class
  \KOMAoptions{parskip=half}}
\makeatother
\usepackage{xcolor}
\usepackage[margin=1in]{geometry}
\usepackage{longtable,booktabs,array}
\usepackage{calc} % for calculating minipage widths
% Correct order of tables after \paragraph or \subparagraph
\usepackage{etoolbox}
\makeatletter
\patchcmd\longtable{\par}{\if@noskipsec\mbox{}\fi\par}{}{}
\makeatother
% Allow footnotes in longtable head/foot
\IfFileExists{footnotehyper.sty}{\usepackage{footnotehyper}}{\usepackage{footnote}}
\makesavenoteenv{longtable}
\usepackage{graphicx}
\makeatletter
\def\maxwidth{\ifdim\Gin@nat@width>\linewidth\linewidth\else\Gin@nat@width\fi}
\def\maxheight{\ifdim\Gin@nat@height>\textheight\textheight\else\Gin@nat@height\fi}
\makeatother
% Scale images if necessary, so that they will not overflow the page
% margins by default, and it is still possible to overwrite the defaults
% using explicit options in \includegraphics[width, height, ...]{}
\setkeys{Gin}{width=\maxwidth,height=\maxheight,keepaspectratio}
% Set default figure placement to htbp
\makeatletter
\def\fps@figure{htbp}
\makeatother
\setlength{\emergencystretch}{3em} % prevent overfull lines
\providecommand{\tightlist}{%
  \setlength{\itemsep}{0pt}\setlength{\parskip}{0pt}}
\setcounter{secnumdepth}{-\maxdimen} % remove section numbering
% definitions for citeproc citations
\NewDocumentCommand\citeproctext{}{}
\NewDocumentCommand\citeproc{mm}{%
  \begingroup\def\citeproctext{#2}\cite{#1}\endgroup}
\makeatletter
 % allow citations to break across lines
 \let\@cite@ofmt\@firstofone
 % avoid brackets around text for \cite:
 \def\@biblabel#1{}
 \def\@cite#1#2{{#1\if@tempswa , #2\fi}}
\makeatother
\newlength{\cslhangindent}
\setlength{\cslhangindent}{1.5em}
\newlength{\csllabelwidth}
\setlength{\csllabelwidth}{3em}
\newenvironment{CSLReferences}[2] % #1 hanging-indent, #2 entry-spacing
 {\begin{list}{}{%
  \setlength{\itemindent}{0pt}
  \setlength{\leftmargin}{0pt}
  \setlength{\parsep}{0pt}
  % turn on hanging indent if param 1 is 1
  \ifodd #1
   \setlength{\leftmargin}{\cslhangindent}
   \setlength{\itemindent}{-1\cslhangindent}
  \fi
  % set entry spacing
  \setlength{\itemsep}{#2\baselineskip}}}
 {\end{list}}
\usepackage{calc}
\newcommand{\CSLBlock}[1]{\hfill\break\parbox[t]{\linewidth}{\strut\ignorespaces#1\strut}}
\newcommand{\CSLLeftMargin}[1]{\parbox[t]{\csllabelwidth}{\strut#1\strut}}
\newcommand{\CSLRightInline}[1]{\parbox[t]{\linewidth - \csllabelwidth}{\strut#1\strut}}
\newcommand{\CSLIndent}[1]{\hspace{\cslhangindent}#1}
\usepackage{setspace}
\doublespacing
\usepackage{lineno}
\linenumbers
\ifLuaTeX
  \usepackage{selnolig}  % disable illegal ligatures
\fi
\usepackage{bookmark}
\IfFileExists{xurl.sty}{\usepackage{xurl}}{} % add URL line breaks if available
\urlstyle{same}
\hypersetup{
  pdftitle={Spatially-explicit predictions using spatial eigenvector maps},
  pdfauthor={Guillaume Guénard; Pierre Legendre},
  hidelinks,
  pdfcreator={LaTeX via pandoc}}

\title{Spatially-explicit predictions using spatial eigenvector maps}
\author{Guillaume Guénard\footnote{Département de sciences biologiques,
  Université de Montréal, PO Box 6128 Centre-Ville Station, Montreal
  Quebec, H3C 3J7 Canada
  \href{mailto:guillaume.guenard@umontreal.ca}{\nolinkurl{guillaume.guenard@umontreal.ca}}} \and Pierre
Legendre\footnote{Département de sciences biologiques, Université de
  Montréal, PO Box 6128 Centre-Ville Station, Montreal Quebec, H3C 3J7
  Canada
  \href{mailto:pierre.legendre@umontreal.ca}{\nolinkurl{pierre.legendre@umontreal.ca}}}}
\date{2024-08-13}

\begin{document}
\maketitle

Running headline: Spatially-explicit predictions

\section{Abstract}\label{abstract}

\begin{enumerate}
\def\labelenumi{\arabic{enumi}.}
\item
  In this paper, we explain how to obtain sets of descriptors of the
  spatial variation, which we call «~predictive Moran's Eigenvector
  Maps~» (pMEM), that can be used to make spatially-explicit predictions
  for any environmental variables, biotic or abiotic. It unites features
  of a method called «~Moran's Eigenvector Maps~» (MEM) and those of
  spatial interpolation, and produces sets of descriptors that can be
  used with any other modelling method, such as regressions, support
  vector machines, regression trees, artificial neural networks, and so
  on. The pMEM are the predictive eigenvectors produced by using a DWF
  in the construction of MEMs. Seven types of pMEM, each associated with
  one of seven different distance weighting functions (DWF), were
  defined and studied.
\item
  We performed a simulation study to determine the power of different
  types of pMEM eigenfunctions at making accurate predictions for
  spatially-structured variables.
\item
  We exemplified the application of the method to the prediction of the
  spatial distribution of \(35\) Oribatid mites living in a peat moss
  (\emph{Sphagnum}) mat on the shore of a Laurentian lake. We also
  provide an \texttt{R} language package called \textbf{pMEM} to make
  calculations easily available to end users.
\item
  The results indicate that anyone of the pMEMs obtained from the
  different distance weighting functions could be the best suited one to
  predict spatial variability in a given data set. Their application to
  the prediction of mite distributions highlights the capability of
  pMEMs for predicting distributions, and for providing
  spatially-explicit estimates of environmental variables that are
  useful for predicting distributions.
\end{enumerate}

\textbf{Key-words}: space, prediction, interpolation, mapping, Moran's I

\subsection{Résumé}\label{ruxe9sumuxe9}

\begin{enumerate}
\def\labelenumi{\arabic{enumi}.}
\tightlist
\item
  Dans cet article, nous expliquons comment obtenir des ensembles de
  descripteurs de la variation spatiale, que nous appelons des ``cartes
  prédictives de vecteurs propres de Moran'' (pMEM), qui peuvent être
  utilisés pour faire des prévisions explicites dans l'espace pour
  n'importe quelle variable environnementale, biotique ou abiotique. Il
  réunit les caractéristiques de la méthode des ``cartes de vecteurs
  propres de Moran (MEM)'' et celles de l'interpolation spatiale, et
  produit des ensembles de descripteurs qui peuvent être utilisés avec
  n'importe quelle autre méthode de modélisation, par exemple les
  méthodes de régression, les machines à vecteurs de support, les arbres
  de régression, les réseaux neuronaux artificiels, etc. Les pMEM sont
  des vecteurs propres prédictifs produits par l'utilisation d'une
  fonction de pondération de la distance dans la construction des MEM.
  Sept types de pMEM, chacun associé à l'une de sept fonctions de
  pondération de la distance (DWF), sont définis et ont été étudiés.
\item
  Nous avons réalisé une étude par simulation pour déterminer la
  puissance de différents types de fonctions propres pMEM à faire des
  prédictions précises pour des variables structurées dans l'espace.
\item
  Nous avons illustré l'application de la méthode à la prédiction de la
  distribution spatiale de 35 espèces d'acariens Oribates vivant dans un
  tapis de tourbe (Sphagnum) sur les rives d'un lac des Laurentides.
  Nous fournissons également un paquet en langage R, appelé pMEM, qui
  rend les calculs facilement accessibles aux utilisateurs.
\item
  Les résultats indiquent que n'importe lequel des ensembles de pMEM
  obtenus à partir des différentes fonctions de pondération de la
  distance pourrait être le mieux adapté pour prédire la variabilité
  spatiale d'un jeu de données particulier. Leur application à la
  prédiction de la distribution des espèces d'acariens met en évidence
  la capacité des pMEM à prédire la distribution des espèces et à
  fournir des estimations spatialement explicites des variables
  environnementales qui sont utiles pour prédire la distribution des
  espèces.
\end{enumerate}

\section{Introduction}\label{introduction}

Spatial analysis hinges on the principle that natural features and
conditions are not distributed haphazardly in space, but are organized
as a consequence of the processes from which they originate (Forman and
Godron 1986; Forman 1995; Legendre 1993). For instance,
spatially-structured geological processes affected the sorting of
minerals in the earth crust, the latter are eroded at various rates by
the action of water, ice, or wind, thereby affecting the distribution of
surface and ground waters which, in turn, are driving the distribution
of microbes, fungi, plants, and animals at various scales in the
landscape. Determining all the relevant natural processes influencing
the distribution of ecosystem components in the landscape is often
undermined by our lack of the necessary data (e.g., Pascoe et al. 2019;
Antunes et al. 2020). Nevertheless, the combined effects of natural
processes are readily visible as spatial structures in the form of
mosaics of gradients, patches of various sizes, shapes and orientations,
and so on. In such circumstances, it is helpful to model feature
distribution directly from their spatial structures instead of relying
on sparsely available environmental descriptors.

Spatial structuring entails that the probability of making a particular
observation at a given location in space is conditional on the values
observed at other points around that location. Consequently, it is
possible to estimate values of a spatially-structured variables at
locations in an area using a set of values of that variable sampled in
the same area. Kriging (Matheron 1962) is an interpolation method that
can be used for that purpose (Legendre and Fortin 1989; Pebesma 2004).
Kriging relies on an estimator of the spatial variation, which is a
function of the pairwise distances between locations, in order to weight
the surrounding observations before averaging. Alternatively, a method
called co-kriging enables one to use data from other observed variables
to help predict the value of a variable of interest (Myers 1984).
Kriging and co-kriging have long been shown to be useful for making
spatially-explicit predictions.

Moran's eigenvector maps (MEM), which were proposed by Dray, Legendre,
and Peres-Neto (2006), are sets of latent descriptors used to represent
spatial variation in models. MEM provide sets of orthogonal (i.e.,
linearly independent) variables generated from the pairwise distances
among the sampling sites, which are calculated from the spatial
distances among the sites, or other types of spatial relationship
matrices, describing, for example, the connectivity among the sites.
Each of these latent variables, which is called a spatial eigenvector
(hereafter referred to as an SEV), has a corresponding eigenvalue, which
is related to, and indicates the spatial scale of, the spatial variation
it describes. SEVs are used as descriptors of spatial variability in any
sort of statistical model suitable to represent single or multiple
random (dependent) variable(s), such as (generalized) linear regression,
regression trees, gradient boosted trees (Mason et al. 1999; Chen and
Guestrin 2016), Bayesian additive regression trees (Chipman, George, and
McCulloch 2010), support vector machines (Cortes and Vapnik 1995),
artificial neural networks (Goodfellow, Bengio, and Courville 2016), and
so on. MEMs get their name from the Moran's index of spatial
autocorrelation (Moran 1950), as there is a simple relationship between
the eigenvalue associated with an SEV and Moran's \(I\) index calculated
for the largest distance class of that SEV. Spatial orthogonal
eigenvectors whose eigenvalues were not strict linear functions of
Moran's \(I\) have also been described, for instances PCNM by Borcard
and Legendre (2002), ISOMAP with anisotropic SEV by Mahecha and
Schmidtlein (2008), and AEM by Blanchet, Legendre, and Borcard (2008),
among other papers (Griffith and Peres-Neto 2006).

\subsection{From discrete to continuous
domain}\label{from-discrete-to-continuous-domain}

Each SEV from an MEM can be regarded as the set of values of an
underlying spatial eigenfunction (hereafter referred to as an SEF) for
the set of sampling sites for which the MEM has been calculated. An SEV
originates from a discrete domain, which is a sample of locations meant
to represent a population of locations, whereas its corresponding SEF
has a continuous domain, and thus bears values for all locations in that
population. To our knowledge, no study has explicitly addressed MEM from
the perspective of continuous SEFs, rather than discrete, point-defined
latent variables (but see, Guénard et al. 2016, 2017; and Guénard and
Legendre 2018, for early applications of this idea). However, this
aspect of MEM is instrumental in using the suite of spatial patterns
described by MEM for spatially-explicit predictive modelling. As such,
predictive spatial modelling using MEM opens the way to applying machine
learning approaches in situations where spatial variation is important
and should be represented in a way that meets the objective of producing
spatially-explicit predictions.

\subsection{Distance weighting}\label{distance-weighting}

Crucially, all MEM-based SEF share the same calculation basis involving
two matrices. The first is a binary connectivity matrix
(\(\mathrm{B} = [b_{i,j}]\)), whose elements take the value \(1\) when
sites \(i\) and \(j\) are linked together, and the value \(0\) when they
are not linked. The second is a spatial weight matrix
(\(\mathrm{A} = [a_{i,j}]\)), whose elements are pairwise weights
calculated from the pairwise between-sites distances using a
distance-weighting function (hereafter referred to as a DWF). The
different types of SEF differ by the nature and specific parameters of
that DWF. For MEM, Dray, Legendre, and Peres-Neto (2006) provided three
DWFs (namely the linear, concave up, and concave down DWFs). Besides
these, it may be useful to explore other suitable DWFs in order to
further our options for SEF. In particular, four of the common variogram
functions used for kriging (namely, the spherical, exponential,
Gaussian, and hole effect variogram functions) can be adapted for use
within the MEM-based predictive SEF framework.

\subsection{Objectives}\label{objectives}

In the present study, we developed the calculations whereby the MEM
framework can be adapted to generate SEF that are suitable for making
predictions (i.e., informed interpolation) for environmental variables
observed in the field, be they abiotic (e.g., temperature, humidity, pH,
pressure) or biotic (e.g., species abundance, density, or diversity). We
also included new DWF derived from common variogram models. We carried
out a simulation study to test their performance at predicting spatial
variation in various situations involving various types of
randomly-generated spatially-structured (Brownian motion) plots, random
sets of sampling locations, and sample sizes for each of the seven DWF
under consideration. Lastly, we exemplified spatial modelling in
practice by modelling the substrate density and water content of the
peat vegetation mat located along the shore of a Canadian Shield bog
lake, and the spatial distribution of \(35\) Oribatid mites living in
that soil.

\section{Materials and methods}\label{materials-and-methods}

\subsection{MEM: Calculation}\label{mem-calculation}

MEM calculation, as defined in Dray, Legendre, and Peres-Neto (2006),
proceeds from the two matrices that we mentioned previously in the
introduction, namely the connectivity \(\mathbf{B}\) and the weights
\(\mathbf{A}\). The next step consists in the Hadamard (element-wise)
product of these two matrices, resulting in a weighted connectivity
matrix (\(\{\mathbf{B*A}\}\)). Matrix \(\mathbf{B}\) has values
\(b_{i,j} = 1\) when any two points \(i\) and \(j\) are connected and
\(b_{i,j} = 0\) otherwise. It can be obtained from a list of edges from
a connectivity graph, such as that derived, for instance, from a
Delauney triangulation, a minimum spanning tree, or simply by
truncation, i.e., by applying a distance threshold to a matrix of
pairwise distances among locations (\([d_{i,j}]\), e.g., a Cartesian or
geodesic two-dimensional space, a three-dimensional Euclidean space, or
a one-dimensional transect), or some other type of connectivity matrix
among the sites. As stated earlier, the spatial weights matrix
\(\mathbf{A}\) may be obtained by transforming the elements of
\([d_{i,j}]\) using a DWF. Following that, matrix \(\{\mathbf{B*A}\}\)
is row- and column-centred to a mean of \(0\) and submitted to
eigenvalue decomposition. By virtue of the centring to \(0\), the
centred weighted connectivity matrix has a rank of at most \(n - 1\),
where \(n\) is the number of different locations. It thus has at most
\(n - 1\) non-zero eigenvalues and eigenvectors. The whole process can
be written in matrix notation as follows:

\[
\mathbf{Q} \{\mathbf{B} * \mathbf{A}\} \mathbf{Q} = \mathbf{U}\mathbf{D}_\lambda\mathbf{U}^\mathsf{T},
\label{eq1}\tag{1}
\]

where
\(\mathbf{Q} = \mathbf{I}_{n} - \frac{1}{n} \mathbf{1}_{n \times n}\) is
the idempotent centring matrix of dimension \(n \times n\)
(\(\mathbf{I}_n\) is an identity matrix of order \(n\) and
\(\mathbf{1}_{n \times n}\) is an \(n \times n\) all-ones matrix),
\(\mathbf{U}\) is a matrix of eigenvectors of dimensions \(n \times k\),
where \(k \leq (n - 1)\), and \(\mathbf{D}_\lambda\) is a diagonal
matrix of (non-zero) eigenvalues. As shown by Jong, Sprenger, and Veen
(2010) there is an algebraic equivalence between these eigenvalues and
the Moran's index (\(I\)) of their corresponding eigenvectors, had
\(\{\mathbf{B*A}\}\) been used during the index calculation. Assuming
the values on the diagonal of \(\{\mathbf{B*A}\}\) to be \(0\), this
equivalence is the following:

\[
I_{\lambda_k} = n \frac{\lambda_k}{\sum_{\forall i,j}b_{i,j}a_{i,j}},
\label{eq2}\tag{2}
\]

Three DWF have been proposed by Dray, Legendre, and Peres-Neto (2006)
(Table~1). It is noteworthy that these functions do not form an
exhaustive set of all possible DWFs; many other such functions can be
developed, which may be suitable for specific questions.

\subsection{Making predictions}\label{making-predictions}

\subsubsection{Distance-weighting
functions}\label{distance-weighting-functions}

In this paper, we are interested in the behaviour of MEM eigenvectors
(SEV) between the sampling locations, in order to assess their potential
as bases for predictive Moran's eigenvector maps (hereafter referred to
as pMEM). While pMEM has a similar purpose as spatial interpolation
methods such as kriging, the former are based on descriptors (i.e., the
column vectors of matrix \(\mathbf{U}\)) rather than on direct
calculations on the raw response data. The SEF used for pMEM are
continuous functions and defined for any location in the space
surrounding the sampling locations. Their values at the sampling
locations are exactly those of the column vectors of \(\mathbf{U}\), but
their values vary at surrounding locations. As such, the SEV are the
expression of the SEF at the sampling locations, whereas the sampling
sites and the surrounding locations define the set of points in space on
which the SEF are mapped. Moreover, the extent and shape of the spatial
structure that the SEFs represent are conditioned by the set of sampling
locations and the distances among them.

There may be a link between the spatial operator (i.e., the DWF) and the
smoothness of the resulting SEF, possibly impacting their adequacy for
representing spatial phenomena. Notably, the smoothness of the SEF in
the vicinity of the sampling locations entails that they are
representative points along continua rather than singularities, around
which sharp spatial shifts may be occurring (See Appendix~II -- Analysis
of SEF shape and smoothness, for an in-depth discussion on that
subject).

For the sake of simplicity, we will restrict the definition of
connectivity to be strictly distance-based and thus, from here,
disregard any graph-based definition. This simplification enables us to
formalize both the connectivity and distance-weighting into single
functions of the distances with parameter \(d_{max}\) acting as a
truncation distance beyond which points are considered non-connected as
follows:

\[
\label{eq3}\tag{3}
w_{i,j} =
  \begin{cases}
    d_{i,j} < d_{max}, f(d_{i,j}; d_{max}, \alpha) \\
    d_{i,j} \geq d_{max}, 0
  \end{cases},
\]

where \(f(d_{i,j}; d_{max}, \alpha)\) is a function of the distance with
a range parameter \(d_{max}\) and a shape parameter \(\alpha\) (see
Appendix~I. Methodological details -- Distance weighting function
derived from the MEM framework, for details about these functions).

These functions take values \(0\) for distances above \(d_{max}\),
thereby involving a threshold in an implicit, distance-based, manner.
For the calculation of pMEM, matrix \(\mathbf{W} = [w_{i,j}]\), can
therefore replace matrix \(\{\mathbf{B*A}\}\) since it involves an
implicit distance threshold \(d \leq d_{max}\). On the other hand, this
definition implies that the value \(1\) is consistently found on the
diagonal of \(\mathbf{W}\), which alters the equivalence between the
eigenvalues and corresponding eigenvector's associated to Moran's index
(\(I\)), which is now calculated as follows:

\[
\label{eq4}\tag{4}
I_{\lambda_k} = n \frac{\lambda_k - 1}{\sum_{\forall i,j} w_{i,j} - n},
\]

Therefore, using a continuous spatial operator has little impact on the
interpretation of the eigenvectors in terms of Moran's index.

\subsubsection{Variogram models}\label{variogram-models}

As stated earlier, the DWFs proposed by Dray, Legendre, and Peres-Neto
(2006) are but a subset of all possible such functions. For this paper,
we propose the addition of four DWFs derived from variogram models
commonly used for kriging (Legendre and Legendre 2012). These functions
are the spherical, exponential, Gaussian, and hole effect DWFs. For
kriging, these variogram functions \(f(d)\) describe how the spatial
variance (\(\gamma(d)\)) increases from a local variance value
(\(\gamma_n\), i.e., the nugget) towards a theoretical maximum variance
value (\(\gamma_s\), i.e., the sill) as the distance increases as
follows:

\[
\label{eq5}\tag{5}
\gamma(d) = \gamma_n + (\gamma_s - \gamma_n)f(d),
\]

where \(f(d)\) is the variogram model function. The distance at which
\(\gamma(d)\) reaches \(\gamma_s\) is called the range of the variogram.
For pMEM, the DWF has a maximum value of \(w_i = 1\) at \(d_i = 0\) and
a minimum value of \(w_i = 0\) at \(d_i = d_{max}\), which corresponds
to the range of the variogram function. Therefore, the variogam-based
DWF are defined as \(w_i = 1 - f(d_i)\) (Table~2).

These functions were studied, alongside the linear, power, and
hyperbolic DWFs presented earlier, and inspired by the ones proposed by
Dray, Legendre, and Peres-Neto (2006), as DWFs for spatial modelling or
plain spatial interpolation using pMEM (Figure~1).

It is noteworthy that parameter \(d_{max}\) in the exponential,
Gaussian, and hole effect DWF do not involve a threshold making
\(w_i=0\) when \(d_i \geq d_{max}\). Also, note that the common
definitions for the exponential and Gaussian DWFs would involve
multiplying \(d_{i,j}/d_{max}\) (or \((d_i/d_{max})^2\)) by \(3\) within
the equations. We regarded that multiplication as superfluous since its
only notable effect is to make the shape of these two DWFs differ more
markedly from that of the other five DWFs; we thus avoided it.

\subsubsection{Spatial eigenfunctions}\label{spatial-eigenfunctions}

One can represent the spatial eigenvectors of the centred weight matrix
by performing an algebraic reorganization of the eigensystem equation
presented earlier (Eq.~\ref{eq1}), as follows:

\begin{align}
\left(\mathbf{I}_n - \frac{1}{n} \mathbf{1}_{n \times n}\right) \mathbf{W} \left(\mathbf{I}_n - \frac{1}{n} \mathbf{1}_{n \times n}\right) &= \mathbf{U} \mathbf{D}_\lambda \mathbf{U}^\mathsf{T} \label{eq6.1}\tag{6.1} \\
\mathbf{I}_n \mathbf{W} \mathbf{I}_n - \frac{1}{n} \mathbf{I}_n \mathbf{W} \mathbf{1}_{n \times n} - \frac{1}{n} \mathbf{1}_{n \times n} \mathbf{W} \mathbf{I}_n + \frac{1}{n^2} \mathbf{1}_{n \times n} \mathbf{W} \mathbf{1}_{n \times n}  &= \mathbf{U}\mathbf{D}_\lambda \mathbf{U}^\mathsf{T} \label{eq6.2}\tag{6.2} \\
\mathbf{W} - \frac{1}{n} \mathbf{W} \mathbf{1}_{n \times n} - \frac{1}{n} \mathbf{1}_{n \times n} \mathbf{W} + \frac{1}{n^2} \mathbf{1}_{n \times n} \mathbf{W} \mathbf{1}_{n \times n}  &= \mathbf{U}\mathbf{D}_\lambda \mathbf{U}^\mathsf{T} \label{eq6.3}\tag{6.3} \\
\left( \mathbf{W} - \frac{1}{n} \mathbf{W} \mathbf{1}_{n \times n} - \frac{1}{n} \mathbf{1}_{n \times n} \mathbf{W} + \frac{1}{n^2} \mathbf{1}_{n \times n} \mathbf{W} \mathbf{1}_{n \times n}  \right) \mathbf{U} \mathbf{D}_{\lambda^{-1}} &= \mathbf{U}  \label{eq6.4}\tag{6.4}
\end{align},

where \(\mathbf{W}\) is the pairwise weight matrix between the
observations. Let \(\mathbf{W}^*\) be the weight matrix calculated for
\(m\) new locations using the same DWF as for \(\mathbf{W}\) and the
distances between the new locations and the original ones found in
\(\mathbf{W}\) (hence, the dimensions of \(\mathbf{W}^*\) are
\(m \times n\)). The values of these new locations on the SEF defined
previously are obtained as follows:

\[
\label{eq7}\tag{7}
\mathbf{U}^* = \left( \mathbf{W}^* - \frac{1}{n} \mathbf{W}^* \mathbf{1}_{n \times n} - \frac{1}{n} \mathbf{1}_{m \times n} \mathbf{W} + \frac{1}{n^2} \mathbf{1}_{m \times n} \mathbf{W} \mathbf{1}_{n \times n}  \right) \mathbf{U} \mathbf{D}_{\lambda^{-1}},
\]

where the matrix of SEF values \(\mathbf{U}^*\) has dimensions
\(m \times k\), with \(k\) being the number of non-zero eigenvalues in
the eigensystem. Using that approach, it is possible to calculate the
values of the SEF at any location, and thus make spatially-explicit
predictions. However, we have yet to provide an assessment of the
adequacy of the seven DWF defined previously for such a purpose.

\subsection{Estimating parameters}\label{estimating-parameters}

The choice of a DWF, as well as the estimation of parameters \(d_{max}\)
and \(\alpha\), can be carried out using different global search
methods. For the present study we propose to select the most suitable
DWF by trying them all, while estimating the most suitable DWF
parameters separately for each of the functions using the directed
evolution approach described by Ardia et al. (2011) and implemented in
\textbf{R} language function \textbf{DEoptim} (Mullen et al. 2011). The
objective criterion to be minimized during the DEoptim global search
procedure was the mean squared prediction error (\(MSE\)). By default,
function \textbf{DEoptim} uses a population size ten times the number of
parameters (i.e., \(20\) individuals in our two-parameter case), and
\(200\) generations.

\subsection{Numerical simulations}\label{numerical-simulations}

We performed a simulation study assessing SEF ability for making
predictions. For that purpose, we generated \(25\) two-dimensional maps.
Each of these maps contained \(5\,184\) points regularly spaced over a
\(72 \times 72\) staggered-row triangular grid pattern with neighbouring
points located at distances \(1\) (in arbitrary spatial units) from one
another. The data were generated at each point of that grid following a
randomly-seeded Wiener process (i.e., Brownian motion) whose
implementation is described in the appendices (Appendix~I.
Methodological details -- Algorithm to generate the spatially-structured
random maps).

To simulate the effect of sampling variation and sample size (\(n\)),
\(25\) sets of \(500\) vertices were randomly selected. From each of
these sets, pairs of subsets of \(n = 10\), \(20\), \(50\), \(100\),
\(200\), and \(500\) were picked as the training data sets, and all
other \(5\,184 - n\) data points were used as the testing data sets.
This procedure resulted in \(3\,750\) simulated data sets (\(25\) maps
\(\times 25\) subsets \(\times 6\) sample sizes). SEF were calculated
from each training simulated data set using each of the seven DWF, for a
grand total sample size of \(26\,250\) (\(3\,750 \times 7\)) trials. For
each of these trials, a \textbf{DEoptim} global search for estimating
parameter values for \(d_{max}\) and \(\alpha\) minimizing the \(MSE\)
was carried out using the default population size and \(50\) generations
(lower than the default in order to mitigate computational time given
the large number of simulations). Values of the lower and upper bounds
for \(d_{max}\) were \(1\) and \(1\,000\), respectively, whereas the
ones for \(\alpha\) were \(0.25\) and \(1.75\), respectively.

Simulations results were analyzed on the basis of the predictions
quality factor \(Q\), which the log ratio of the mean square deviation
\(MSD\) and the mean squared error \(MSE\), whereas the coefficient of
prediction (\(P^2\)) was used to display the results (see Appendix~I.
Methodological details -- Calculations on the simulation results, for a
justification of using \(Q\) and for \(P^2\) and details on their
calculation).

Simulation results were analyzed using the analysis of variance (ANOVA).
Two such analyses were performed. A first ANOVA was carried out on all
\(26\,250\) trials using four variables, one quantitative: the base-10
logarithm of the sample size (\(\log_{10} n\)), and three qualitative
(or factors): \(DWF\), \(Map\), and \(Sample\), as well as their six
second-order interaction and four third-interaction terms. A second
restrained ANOVA was performed on the subset of the best-DWF trial for
each of the \(3\,750\) simulated data sets. The latter was a
three-variable design with variables: \(\log_{10} n\), \(Map\), and
\(Sample\), and their four second-order and single third-order
interaction terms.

\subsection{Application example}\label{application-example}

\subsubsection{Data set}\label{data-set}

The SEF prediction approach described in the present study was applied
to a well-studied data example. The chosen data set involves the
distribution of \(35\) taxa of Oribatid mite (class: Arachnida) in a
peat bog mat located on the shore of Lac Geai, a small lake located on
the territory of «~Station de Biologie des Laurentides, Université de
Montréal~», in the conurbation of St.~Hippolyte, Quebec, Canada. This
data set was first described by Borcard and Legendre (1994) and various
copies are available, notably from \texttt{R} packages \textbf{ade4}
(\texttt{oribatid}), \textbf{codep} (\texttt{mite}), and \textbf{vegan}
(\texttt{mite}) as well as from the Borcard, Gillet, and Legendre (2018)
book. Sampling was carried out in June of 1989.

The sampling area was \(10\,\mathrm{m}\) long by \(2.6\,\mathrm{m}\)
wide, with the long axis stretching from the forest to the open water of
the lake. The coordinates of its centre were approximately
(\(45.99549\), \(-73.99370\)). Further details on the lake, its water,
and its surroundings are found in Borcard and Legendre (1994).

Core samples of peat were taken and the Oribatid mites inhabiting them
were extracted, sorted, identified, and classified into \(35\)
morphospecies and genera. These taxa are chiefly based on morphology,
since relatively little was known about the ecology and physiology of
these small animals. The Oribatid community structure was analyzed using
a principal component analysis (PCA) of the Hellinger-transformed
abundance data (Legendre and Gallagher 2001), keeping the first two
principal components.

In addition to the Oribatid mite counts, the data set includes six
environmental predictors, namely, (1) the substrate density
(quantitative; the mass of an unpacked volume dry substrate,
\(\mathrm{g \cdot L^{-1}}\)), (2) the water content (quantitative; the
mass of water by volume of wet substrate, \(\mathrm{g \cdot L^{-1}}\)),
(3) the substrate type (qualitative; represented by six non mutually
exclusive binary-coded classes: «~Sphagn1~», «~Sphagn2~», «~Sphagn3~»,
«~Sphagn4~», «~Litter~», «~Bare peat~»), shrub density
(semi-quantitative; three levels: «~None~», «~Few~», «~Many~»),
topography (qualitative; two mutually exclusive classes: «~Hummock~» and
«~Blanket~»), and a binary variable indicating flooded areas. This last
variable was obtained from the maps in Borcard and Legendre (1994)
(their figure~1) and is not available in the data sets in \texttt{R}
packages \textbf{ade4}, \textbf{codep}, and \textbf{vegan}.

We assembled the data points into a single point geometry stored as a
geopackage file and added polygon geometries for the substrate type,
shrub density, topography, and flooded areas, which we outlined manually
at a resolution of roughly \(0.01\,\mathrm{m}\) from the three images
obtained from figure~1 in Borcard and Legendre (1994) using software
QGIS \url{https://qgis.org} (Figure~2). The Oribatid mites and
environmental data matrix contains the variables on which spatial
modelling will be carried out in this example.

\subsubsection{Modelling}\label{modelling}

Two continuous environmental variables, namely substrate density and
water content, were not available from geographic information layers,
but measurements had been taken at the sampling point locations. To be
able to use them for predicting the density of the different mites at
any location over the sampling area, a continuous map of these variables
was needed. We took this need as an opportunity to illustrate
single-variable prediction using SEF exclusively. We began be generating
a point grid over the sampling area with a resolution of
\(5\,\mathrm{cm}\). This grid was used as a basis for generating GIS
rasters for the different variables involved in this example. Variables
substrate density and water content were modelled using an
\(L_1\)-regularized (LASSO) linear regression model calculated using
\texttt{R} package \textbf{glmnet} (Friedman, Tibshirani, and Hastie
2010) using the Gaussian family of Generalized Linear Models (GLM) and
predicted values were computed over the grid points. Values of
parameters \(d_{max}\) and \(\alpha\) were estimated by \textbf{DEoptim}
(default parameters), using \(d_{max}\) values between \(1\) and
\(5\,\mathrm{m}\) and \(\alpha\) values between \(0.15\) and \(1.85\).
Seven cross-validation folds were used for estimating the predictive
power. Assignment to cross-validation folds was carried out in a
systematic manner following the order in which the data appear in the
data set by selecting data points with indices \(i + 7*j\) where \(i\)
is the cross-validation fold (\(1\)--\(7\)) and \(j = 0, 1, 2, ..., 9\).

Variables \emph{substrate type}, \emph{shrub density},
\emph{topography}, and \emph{flooded} were available directly from the
polygon geometries. Variable \emph{substrate type} was a set of non
mutually exclusive classes, since cores had sometimes purposefully been
taken at the boundaries of areas with different substrates in order to
study ecological transitions. Therefore, this variable was available as
a six-column matrix of binary (or dummy) variables rather than as a
single factor with mutually exclusive levels. Each element of that
binary matrix was divided by the sum of the row in order to make all the
rows of the resulting transformed matrix sum to \(1\). This treatment
made the effects of the substrate types additive. Variable \emph{shrub
density} was semi-quantitative and treated using polynomial contrasts,
whereas variable \emph{topography}, which has two levels was transformed
into a binary variable and centred to a mean value of \(0\). Finally,
variable \emph{flooded} was used as is.

We modelled Oribatid mite distributions from the individual count data
using a Poisson-family \(L_1\)-regularized multivariate generalized
regression model (GLM), which was also calculated using the \texttt{R}
package \textbf{glmnet} (Friedman, Tibshirani, and Hastie 2010; Tay,
Narasimhan, and Hastie 2023). We used customized \texttt{R} language
code to allow \textbf{glmnet} to handle a multivariate response (i.e.,
the 35 mites) since the package does not support multivariate models
natively. The model's quality of fit was estimated separately for each
mite using the likelihood-based \(R^2\) coefficient for the
Poisson-family of GLM proposed by Guénard et al. (2017). Values of
parameters \(d_{max}\) and \(\alpha\) were estimated by \textbf{DEoptim}
under identical conditions as for the aforementioned substrate density
and water content models.

\section{Results}\label{results}

\subsection{Simulations}\label{simulations}

The analysis of variance computed over all simulation results reveals
that \(Q_{pred}\) was most affected by the sample size of the training
set (\(\log_{10}n\), Table~3). This result was expected; it is well know
that the potential of a model at generalizing its target data is a
function of the sample size of its training data, as it is a consequence
of Hoeffding's inequality (Hoeffding 1963). The second most important
factor was \(Map\), which showed that maps generated during the
simulation had various levels of predictability by spatial modelling.
The third factor was \(DWF\), followed by \(Sample\) (see Appendix~III
-- Figures~AIII~1--3 for details on these results). The marginal effects
of these factors were all statistically significant. All but one of the
second-order interaction terms were also statistically significant, the
notable exception being interaction term \(DWF \times Sample\). All
second order interaction terms involving \(\log_{10}n\) and \(Map\) were
statistically significant. Two of the four third-order interaction terms
were statistically significant: interaction term
\(\log_{10}n \times DWF \times Map\), indicating that the manner by
which \(\log_{10}n\) affects \(Q_{pred}\) varies among various DWF-Map
combinations, and interaction term
\(\log_{10}n \times Map \times Sample\), indicating that the effect of
\(\log_{10}n\) is also modulated in various ways among the Map-Sample
combinations.

The distance weighting function that was the most frequently associated
with the best model was the power function (\(1633\) instances,
\(43.5\%\)), followed by the Gaussian, (\(684\), \(18.2\%\)), the hole
effect (\(499\), \(13.3\%\)), the exponential, (\(359\), \(9.6\%\)), the
hyperbolic (\(320\), \(8.5\%\)), the spherical (\(147\), \(3.9\%\)),
and, finally, the linear DWF (\(108\), \(2.9\%\)). During the
simulations, the \(Q_{pred}\) of the best-DWF models was also mainly
affected by the sample size (Table~4). The mean \(P^2\) was 0.2959 when
\(n = 10\) (\(Q_{pred} = 0.1524\)), and increased to 0.4538 when
\(n = 20\) (\(Q_{pred} = 0.2627\)), to 0.6096 when \(n = 50\)
(\(Q_{pred} = 0.4085\)), to 0.6972 when \(n = 100\)
(\(Q_{pred} = 0.5188\)), to 0.7651 when \(n = 200\)
(\(Q_{pred} = 0.6292\)), and finally to 0.8321 when \(n = 500\)
(\(Q_{pred} = 0.775\)).

The \(Q_{pred}\) of the best-DWF models also varied among the maps and,
but to a much lesser extent, among the subsets. The significant
among-map variation in the \(Q_{pred}\) entails that some of the maps
are more or less predictable than others as a consequence of their
random origin from sets of sporadically spread initial points (See
appendix~III -- Supplementary figures -- Simulation results).
Interaction term \(\log_{10}n \times Map\) was also statistically
significant, indicating that an increase in the size of the training
sample improves predictions for some of the maps more than for some
others.

The among-sample variation of the \(Q_{pred}\) was smaller than that of
\(Map\), and interaction term \(\log_{10}n \times Sample\) was also
significant (See appendix~III -- Supplementary figures -- Simulation
results). It thus appears that some of the randomly-generated training
samples were more suitable than some others to properly sample the maps,
and that this suitability was increased in different ways as the sample
size was increased.

Finally, interaction terms \(Map \times Sample\) and
\(\log_{10}n \times Map \times Sample\) were also statistically
significant, highlighting that the different random training samples had
varying suitability at representing the different maps, and that this
suitability also increased in different ways with increasing training
sample size.

\subsection{Oribatid mite example}\label{oribatid-mite-example}

The best subordinate model predicting substrate density was found to use
the power DWF (Appendix I -- Eq.~A2) with a range of
\(d_{max} = 1.14\,\mathrm{m}\) and a shape parameter value of
\(\alpha = 0.67\). This model was made of six SEF; the square root of
the mean squared error (\emph{RMS}) was \(11.3\,\mathrm{g~L^{-1}}\)
(\(P^2 = 0.088\); Figure~3). This model was thus only slightly better
than taking the mean value substrate density
(\(39.28\,\mathrm{g~L^{-1}}\)) as the predicted value. For the water
content model, the best DWF was the Gaussian DWF (Eq.~T2~3 from Table~2)
with a range of \(d_{max} = 1.12\,\mathrm{m}\), comprising \(11\) SEF;
the \emph{RMSE} was \(122.5\,\mathrm{g~L^{-1}}\) (\(P^2 = 0.25\)).

The best DWF for predicting Oribatid mite distribution was the power DWF
(Appendix~I -- Eq.~A2), with a range of \(d_{max} = 2.34\,\mathrm{m}\),
a shape parameter value of \(\alpha = 1.68\), and deviance value
(\(-2\log L\)) of \(4.137\). The model's likelihood-based \(R^2\) varied
from \(0.073\) for mite \emph{Hyporufu} to \(0.878\) for mite
\emph{Limncfci} (median: \(0.548\); Appendix~II -- Table~A-II~1). The
ability of the model to predict mite counts was proportional to the mean
abundance of the mites in the sampling area (\(F_{1,33} = 18.32\),
\(P < 0.001\); with log-transformed mean abundance and predictability
estimated as \(Q = -\log_{10} (1 - R^2)\)). For instance, the expected
\(R^2\) is \(0.357\) for a mean count of
\(0.157~\mathrm{ind.~core^{-1}}\) (the minimum value observed),
\(0.574\) for a mean count of \(1~\mathrm{ind.~core^{-1}}\), \(0.745\)
for a mean count of \(10~\mathrm{ind.~core^{-1}}\), and \(0.807\) for a
mean count of \(35.26~\mathrm{ind.~core^{-1}}\) (the maximum value
observed). Also consistent with this result is the observation that
mites absent from a large number of sites (e.g., \emph{Hyporufu}, which
is absent from \(60\) of the \(70\) sites) tend to have a small \(R^2\)
compared to mites present in many sites (e.g., \emph{Limncfci}, which is
absent from only \(15\) of the \(70\) sites).

\subsubsection{Community structure}\label{community-structure}

The two axes of the PCA carried out on the transformed response data
matrix accounted for approximately \(25\%\) of the variance of the data
matrix (Figure~2). The first PCA axis was driven chiefly by the
preponderance of \emph{Tectvela} \emph{Oppiniva}, and \emph{Suctobsp},
which are associated with negative loading, with respect to that of
\emph{Limncfci}, \emph{Limncfru}, and, to a lesser extent,
\emph{Trhyposp} and \emph{Trimalsp}, which are associated with positive
loading. The second PCA axis was driven by the preponderance of
\emph{Limncfru}, \emph{Hoplcfpa}, and \emph{Suctobsp}, which are
associated with negative loading, with respect to that of
\emph{Limncfci}, \emph{Trhyposp}, and \emph{Tectvela}, which are
associated with positive loading.

The components of the Oribatid community structure described by the PCA
axes followed their own particular distribution spatial patterns
(Figure~4). For the first PCA axis, large negative values were observed
close to the forest line, at a distance of approximately
\(1\,\mathrm{m}\) from the lower end of the plot, whereas large positive
values were observed near the waterline at a distance of approximately
\(1\,\mathrm{m}\) from it. The most extreme values of the second PCA
axis (positive) were observed close the the forest and in and around the
flooded areas. Negative second PCA axis loading values were observed on
the right of the map at around a third of the distance from the
waterline and forest line.

\section{Discussion}\label{discussion}

In the present study, we developed the predictive Moran's Eigenvector
Maps, a computational framework for making spatially-explicit spatial
predictions at arbitrary locations about sampling points bearing known
values. This goal is similar to that of common spatial interpolation
methods such as kriging. However, whereas interpolation methods are
non-parametric and thus based on the direct involvement of the data
points, pMEM is a parametric method involving explanatory descriptors.
That property entails that pMEM is a method that does not provide direct
interpolation estimates of the variable it seeks to estimate. Instead,
it provides descriptors, in the form of SEF, to be used later during
analyses and model development. These descriptors are usable as is
(e.g., when predicting substrate density or water content in the mite
example) or in combination with additional descriptors (e.g., when
predicting Oribatid mite distributions in our example). Furthermore, any
suitable model estimation approach can be used during the subsequent
steps of the modelling workflow (e.g., an \(L_1\) regularized
generalized linear model in the oribatid mite examples). Besides the
more common linear model estimation methods such as the one we used in
the example, alternative machine learning methods can also be used.
These methods include regression trees, gradient boosted trees (Mason et
al. 1999; Chen and Guestrin 2016), Bayesian additive regression trees
(Chipman, George, and McCulloch 2010), support vector machines (Cortes
and Vapnik 1995), artificial neural networks (Goodfellow, Bengio, and
Courville 2016), among others. In machine learning parlance, pMEM is
referred to as a «~feature engineering~» approach (Chollet and Allaire
2018). This preliminary step involves the introduction of a numerical
representation of the spatial coordinates in the model, in the form of
latent variables. The addition of this numerical representation helps
the model in modelling the response(s) on the basis of estimated spatial
variation patterns.

Since pMEM involve descriptors, error estimation on the predictions is
handled by the method that uses them for modelling. For instance, the
handling of prediction error is well-established for multiple linear
regression (but see Zhang (1993) for a caveat on using that approach).
At the price of more computational power, cross-validation or other
random sampling approaches (e.g., bootstrap, jackknife) can be used to
obtain numerical estimates of the prediction error for virtually any
modelling method. The details about the estimation of prediction error
belongs to the particular method using the pMEM eigenfunctions and are
thus outside the scope of the present study.

The simulation study we performed indicates that any of the DWF may at
times yield sets of SEF that were the most appropriate to model the
simulated data, which were samples from two-dimensional maps generated
by Brownian motion simulations. When applied to real data, the three
models built involved SEF from two DWF: the power DWF and the Gaussian
DWF. These observations indicate that SEF with different orders of
continuity may be equally suitable for spatial modelling and that having
multiple DWF is a beneficial aspect of the pMEM toolbox, as it is
presently developed. Actually, other DWF besides the ones described in
the present study may be proposed in future developments of the pMEM
method.

Simulation results indicated that pMEM were able to model and predict
spatially structured variables with various degrees of success,
depending primarily on the sample size and secondarily on a suite of
other factors related to sampling and DWF selection, albeit to a lesser
extent (Table~3). Simulation results highlighted that the data
generation procedure was also successful at producing maps with various
degrees of predictability using pMEM. Some of the DWF were more often
selected than others as the best-suited one for a given set of
conditions (in terms of spatial context, sample, and so on). For
instance, the power DWF was the most commonly selected and the linear
DWF was the least commonly selected, yet every DWF was found to be the
adequate at making spatially-explicit predictions on given
\(Map \times Sample\) combinations. On the one hand, picking the most
suitable DWF was not as important for spatial predictability as the
sample size, and its effect was relatively small with respect to the
among-map variability, yet more important than the among-sample
variability. On the other hand, choosing the most suitable DWF incurs no
supplementary cost, unlike increasing the sample size, or altering the
sampling approach.

The present study exemplified the use of pMEM using a modest-sized data
set involving \(70\) observations. Using SEF-only models and regularized
regressions, we were nevertheless able to predict the spatial
distributions of two environmental variable, the substrate's density and
water content, with some success (\(P^2>0\)). Then, using complex models
involving environmental variables, we have been able to predict the
distribution of \(35\) mites with various degrees of success. For
instance, substrate density was predicted with a modest accuracy
(\(P^2\) of \(0.088\)), with an \(RMSE\) of \(11.3\,\mathrm{g~L^{-1}}\),
which was only slightly above the variable's standard deviation
(\(11.9\,\mathrm{g~L^{-1}}\)). Substrate water content was slightly more
accurately predicted (\(P^2=0.25\)), with an \(RMSE\) of
\(122.5\,\mathrm{g~L^{-1}}\) for a standard deviation of
\(142.4\,\mathrm{g~L^{-1}}\). Model accuracy for mite distribution was
mainly influenced by the observed counts, with \(P^2\) values from a
minimum of \(0.073\) to a maximum of \(0.878\). This result was not
unexpected as the rare mites were absent from most cores and only found
at low frequencies in a few other cores, thereby making the
determination of their preferred conditions more uncertain. On the other
hand, the more prevalent mites were observed in most of the cores with
low to higher frequencies, a situation that makes it easier to determine
the preferred conditions sought after by the mites, provided that
relevant descriptors are available.

The computation of the pMEM relies on square matrices for storing
distances and the weights and on eigenvalue decomposition, which is a
computationally demanding method. While it is not a problem for small
data sets such as the ones shown in the present study, requirements in
term of computer memory storage and computation time become prohibitive
on large data set (a few thousand data points) even for state of the art
computer systems. A straightforward solution to adapting pMEM to large
data set would be to consider using the pairwise distances between the
\(n\) data points and a set of \(k\) representative spatial kernels
disseminated over the study area. The resulting \(n \times k\)
rectangular distance matrix could be transformed into a spatial weights
matrix, submitted to centering and then to singular value decomposition
(SVD). By choosing a parsimonious number of kernels, the kernel-based
pMEM thus obtained would remain applicable to large data sets (in the
tens or hundreds of thousand of data points). Given that \(k\) would be
much smaller than \(n\), the number of SEF would be equal to \(k\) (or,
perhaps, slightly smaller). This property might also help in simplifying
model building to some extent. That proposal opens other matters that we
did not have to ponder while studying pMEM. For instance, the approach
for choosing the number of kernels and their locations would have to be
considered (e.g., using medoids \emph{vs} centroids). Also, the linear
algebra linking the Moran's I index to the SEF thus defined would need
to be demonstrated, since that link is helpful in interpreting the
spatial scale associated with the SEF. These matters, and possibly other
unanticipated matters that may likely spring up while developing
kernel-based pMEM, are clearly beyond the scope of the present study.

The pMEM framework may be useful for other purposes besides our
resolutely machine-learning oriented objective of using it for making
spatially-explicit predictions. For instance, one may consider using it
to correct the confounding effect of spatial autocorrelation when
carrying out statistical inference testing. However, it is worth
recalling that pMEM are identical to MEM when only considering the
sampling points. To what extent the four variogram-based DWF we
introduced would improve MEM performance in correcting spatial
confounding will remain an unanswered questions until a thorough
simulation study addressing that matter is carried out. In the meantime,
we consider it safer to assume that actual knowledge about spatial
confounding still holds, and thus would direct the reader to the
literature on that particular subject matter (Thaden and Kneib 2018;
Dupont, Wood, and Augustin 2022; Marques, Kneib, and Klein 2022; Mäkinen
et al. 2022).

We are hoping that the findings highlighted in the present study will
entice scientists to use pMEM to model spatial variation and for making
predictions, and software developers to expand the implementation of the
approaches to other computer languages and software.

\section{Conflict of interest
statement}\label{conflict-of-interest-statement}

We declare no conflict of interest.

\section{Data availability}\label{data-availability}

An R package called pMEM and all the data used for this study (computer
simulations, example calculations, Appendices) are available through the
following link: \href{https://doi.org/10.5281/zenodo.13311594}{DOI:
10.5281/zenodo.13311594}.

\pagebreak

\section{References}\label{references}

\phantomsection\label{refs}
\begin{CSLReferences}{1}{0}
\bibitem[\citeproctext]{ref-antunes_quantitative_2020}
Antunes, N., W. Schiefenhövel, F. d'Errico, W. E. Banks, and M.
Vanhaeren. 2020. {``Quantitative Methods Demonstrate That Environment
Alone Is an Insufficient Predictor of Present-Day Language Distributions
in {New} {Guinea}.''} \emph{PLOS ONE} 15 (10): e0239359.
\url{https://doi.org/10.1371/journal.pone.0239359}.

\bibitem[\citeproctext]{ref-Ardia-RJ-2011-005}
Ardia, D., K. Boudt, P. Carl, K. M. Mullen, and B. G. Peterson. 2011.
{``Differential Evolution with {DEoptim}.''} \emph{{The R Journal}} 3
(1): 27--34. \url{https://doi.org/10.32614/RJ-2011-005}.

\bibitem[\citeproctext]{ref-blanchet_modelling_2008}
Blanchet, F. G., P. Legendre, and D. Borcard. 2008. {``Modelling
Directional Spatial Processes in Ecological Data.''} \emph{Ecol. Model.}
215: 325--36. \url{https://doi.org/10.1016/j.ecolmodel.2008.04.001}.

\bibitem[\citeproctext]{ref-borcard_numerical_2018}
Borcard, D., F. Gillet, and P. Legendre. 2018. \emph{Numerical Ecology
with {R}, 2\textsuperscript{nd} Edition}. Springer International
Publishing AG.

\bibitem[\citeproctext]{ref-borcard_environmental_1994}
Borcard, D., and P. Legendre. 1994. {``Environmental Control and Spatial
Structure in Ecological Communities: {A}n Example Using {O}ribatid Mites
({A}cari, {O}ribatei).''} \emph{Environ. Ecol. Stat.} 1 (1): 37--61.
\url{https://doi.org/10.1007/BF00714196}.

\bibitem[\citeproctext]{ref-borcard_all-scale_2002}
---------. 2002. {``All-Scale Spatial Analysis of Ecological Data by
Means of Principal Coordinates of Neighbour Matrices.''} \emph{Ecol.
Model.} 153: 51--68.
\url{https://doi.org/10.1016/S0304-3800(01)00501-4}.

\bibitem[\citeproctext]{ref-chen_xgboost_2016}
Chen, T., and C. Guestrin. 2016. {``{XGBoost}: A Scalable Tree Boosting
System.''} In \emph{Proceedings of the 22nd {ACM} {SIGKDD} International
Conference on Knowledge Discovery and Data Mining}, 785--94. {KDD} '16.
New York, NY, USA: Association for Computing Machinery.

\bibitem[\citeproctext]{ref-chipman_bart_2010}
Chipman, H. A., E. I. George, and R. E. McCulloch. 2010. {``{BART}:
{Bayesian} Additive Regression Trees.''} \emph{Ann. Appl. Stat.} 4 (1):
266--98. \url{https://doi.org/10.1214/09-AOAS285}.

\bibitem[\citeproctext]{ref-chollet_deep_2018}
Chollet, F., and J. J. Allaire. 2018. \emph{Deep Learning with {R}}.
Manning Publications.

\bibitem[\citeproctext]{ref-cortes_support-vector_1995}
Cortes, C., and V. Vapnik. 1995. {``Support-Vector Networks.''}
\emph{Mach. Learn.} 20 (3): 273--97.
\url{https://doi.org/10.1007/BF00994018}.

\bibitem[\citeproctext]{ref-dray_spatial_2006}
Dray, S., P. Legendre, and P. Peres-Neto. 2006. {``Spatial Modelling: A
Comprehensive Framework for Principal Coordinate Analysis of Neighbour
Matrices ({PCNM}).''} \emph{Ecol. Modelling} 196: 483--93.

\bibitem[\citeproctext]{ref-Dupont_Spatial_2022}
Dupont, E., S. N. Wood, and N. H. Augustin. 2022. {``Spatial+: A Novel
Approach to Spatial Confounding.''} \emph{{Biometrics}} 78 (4):
1279--90.

\bibitem[\citeproctext]{ref-forman_land_1995}
Forman, R. T. T. 1995. \emph{Land Mosaics: The Ecology of Landscapes and
Regions}. Cambridge, UK.: Cambridge University Press.

\bibitem[\citeproctext]{ref-forman_landscape_1986}
Forman, R. T. T., and M. Godron. 1986. \emph{Landscape Ecology}. New
York, NY, USA.: John Wiley; Sons, Inc.

\bibitem[\citeproctext]{ref-Friedman2010GlmNet}
Friedman, J., R. Tibshirani, and T. Hastie. 2010. {``Regularization
Paths for Generalized Linear Models via Coordinate Descent.''} \emph{J.
Stat. Softw.} 33 (1): 1--22.
\url{https://doi.org/10.18637/jss.v033.i01}.

\bibitem[\citeproctext]{ref-Goodfellow-et-al-2016}
Goodfellow, I., Y. Bengio, and A. Courville. 2016. \emph{Deep Learning}.
MIT Press.

\bibitem[\citeproctext]{ref-griffith_spatial_2006}
Griffith, D. A., and P. R. Peres-Neto. 2006. {``Spatial Modeling in
Ecology: The Flexibility of Eigenfunction Spatial Analyses.''}
\emph{Ecology} 87: 2603--13.

\bibitem[\citeproctext]{ref-guenard_spatially-explicit_2016}
Guénard, G., G. Lanthier, S. Harvey‐Lavoie, C. J. Macnaughton, C. Senay,
M. Lapointe, P. Legendre, and D. Boisclair. 2016. {``A
Spatially-Explicit Assessment of the Fish Population Response to Flow
Management in a Heterogeneous Landscape.''} \emph{Ecosphere} 7 (5):
e01252.

\bibitem[\citeproctext]{ref-guenard_modelling_2017}
---------. 2017. {``Modelling Habitat Distributions for Multiple Species
Using Phylogenetics.''} \emph{Ecography} 40 (9): 1088--97.

\bibitem[\citeproctext]{ref-guenard_bringing_2018}
Guénard, G., and P. Legendre. 2018. {``Bringing Multivariate Support to
Multiscale Codependence Analysis: Assessing the Drivers of Community
Structure Across Spatial Scales.''} \emph{Meth. Ecol. Evol.} 9:
292--304. \url{https://doi.org/10.1111/2041-210X.12864}.

\bibitem[\citeproctext]{ref-hoeffding_probability_1963}
Hoeffding, W. 1963. {``Probability Inequalities for Sums of Bounded
Random Variables.''} \emph{J. Am. Stat. Assoc.} 58 (301): 13--30.
\url{https://doi.org/10.1080/01621459.1963.10500830}.

\bibitem[\citeproctext]{ref-jong_extreme_2010}
Jong, P., C. Sprenger, and F. Veen. 2010. {``On Extreme Values of
{M}oran's {I} and {G}eary's {c}.''} \emph{Geogr. Anal.} 16 (1): 17--24.
\url{https://doi.org/10.1111/j.1538-4632.1984.tb00797.x}.

\bibitem[\citeproctext]{ref-Legendre_Spatial_1993}
Legendre, P. 1993. {``Spatial Autocorrelation: Trouble or New
Paradigm?''} \emph{Ecology}, no. 6: 1659--73.
\url{https://doi.org/10.2307/1939924}.

\bibitem[\citeproctext]{ref-legendre_spatial_1989}
Legendre, P., and M. J. Fortin. 1989. {``Spatial Pattern and Ecological
Analysis.''} \emph{Vegetatio} 80 (2): 107--38.

\bibitem[\citeproctext]{ref-legendre_ecologically_2001}
Legendre, P., and E. D. Gallagher. 2001. {``Ecologically Meaningful
Transformations for Ordination of Species Data.''} \emph{Oecologia} 129:
271--80.

\bibitem[\citeproctext]{ref-legendre_numerical_2012}
Legendre, P., and L. Legendre. 2012. \emph{Numerical Ecology, 3rd
{E}nglish Edition}. Amsterdam, The Netherlands: Elsevier Science B.V.

\bibitem[\citeproctext]{ref-mahecha_revealing_2008}
Mahecha, M. D., and S. Schmidtlein. 2008. {``Revealing Biogeographical
Patterns by Nonlinear Ordinations and Derived Anisotropic Spatial
Filters.''} \emph{Global Ecology and Biogeography} 17 (2): 284--96.
\url{https://doi.org/10.1111/j.1466-8238.2007.00368.x}.

\bibitem[\citeproctext]{ref-Makinen_Spatial_2022}
Mäkinen, J., E. Numminen, P. Niittynen, M. Luoto, and J. Vanhatalo.
2022. {``Spatial Confounding in {Bayesian} Species Distribution
Modeling.''} \emph{{Ecography}} 33 (11): e06183.

\bibitem[\citeproctext]{ref-Marques_Mitigating_2022}
Marques, I., T. Kneib, and N. Klein. 2022. {``Mitigating Spatial
Confounding by Explicitly Correlating Gaussian Random Fields.''}
\emph{{Environmetrics}} 33 (5): e2727.

\bibitem[\citeproctext]{ref-mason_boosting_1999}
Mason, L., J. Baxter, P. Bartlett, and M. Frean. 1999. {``Boosting
Algorithms as Gradient Descent.''} In \emph{Advances in Neural
Information Processing Systems}, MIT Press. Vol. 12. Boston, MA, USA.

\bibitem[\citeproctext]{ref-Matheron_1962}
Matheron, G. 1962. \emph{Traité de Géostatistique Appliquée. Tomes {I}
Et {II}}. Paris: Éditions Technip.

\bibitem[\citeproctext]{ref-moran_notes_1950}
Moran, P. A. P. 1950. {``Notes on Continuous Stochastic Phenomena.''}
\emph{Biometrika} 37 (1/2): 17--23.
\url{https://doi.org/10.2307/2332142}.

\bibitem[\citeproctext]{ref-Mullen-JSS-40-6-1-26}
Mullen, K. M., D. Ardia, D. Gil, D. Windover, and J. Cline. 2011.
{``{DEoptim}: An {R} Package for Global Optimization by Differential
Evolution.''} \emph{{J. Stat. Soft.}} 40 (6): 1--26.
\url{https://doi.org/10.18637/jss.v040.i06}.

\bibitem[\citeproctext]{ref-myers_co-kriging_1984}
Myers, D. E. 1984. {``Co-Kriging --- New Developments.''} In
\emph{Geostatistics for Natural Resources Characterization: Part 1},
edited by G. Verly, M. David, A. G. Journel, and A. Marechal, 295--305.
Dordrecht: Springer Netherlands.
\url{https://doi.org/10.1007/978-94-009-3699-7_18}.

\bibitem[\citeproctext]{ref-pascoe_lack_2019}
Pascoe, E. L., S. Pareeth, D. Rocchini, and M. Marcantonio. 2019. {``A
Lack of {`Environmental Earth Data'} at the Microhabitat Scale Impacts
Efforts to Control Invasive {A}rthropods That Vector Pathogens.''}
\emph{Data} 4 (4): 133. \url{https://doi.org/10.3390/data4040133}.

\bibitem[\citeproctext]{ref-pebesma_multivariable_2004}
Pebesma, E. J. 2004. {``Multivariable Geostatistics in {S}: The Gstat
Package.''} \emph{Comput. Geosci.} 30 (7): 683--91.
\url{https://doi.org/10.1016/j.cageo.2004.03.012}.

\bibitem[\citeproctext]{ref-Tay2023GlmNet}
Tay, J. K., B. Narasimhan, and T. Hastie. 2023. {``Elastic Net
Regularization Paths for All Generalized Linear Models.''} \emph{J.
Stat. Softw.} 106 (1): 1--31.
\url{https://doi.org/10.18637/jss.v106.i01}.

\bibitem[\citeproctext]{ref-Thaden_Structural_2018}
Thaden, H., and T. Kneib. 2018. {``Structural Equation Models for
Dealing with Spatial Confounding.''} \emph{Am. Stat.} 72 (3): 239--52.

\bibitem[\citeproctext]{ref-zhang_estimation_1993}
Zhang, P. 1993. {``On the Estimation of Prediction Errors in Linear
Regression Models.''} \emph{{Ann. Inst. Stat. Math.}} 45 (1): 105--11.
\url{https://doi.org/10.1007/BF00773671}.

\end{CSLReferences}

\section{Figures and tables}\label{figures-and-tables}

Figure 1. Distance-weighting functions whose potential for
spatially-explicit modelling was assessed in this study.

\begin{center}\rule{0.5\linewidth}{0.5pt}\end{center}

Figure 2. Maps of substrate types, shrub density, and topography
outlined from Borcard and Legendre (1994, their Figure 1), together with
a principal component analysis (PCA) biplot showing the sampling sites
(markers) and Oribatid mites (arrows). The maps also feature the
sampling points (black dots), open waters (blue area), and flooded areas
(circumscribed by dotted curves). The labels at the tip of the PCA
biplot arrows are the names of the eight mites with the largest axis
loadings in their vicinity. The two PCA axes represent approximatly a
fourth of the total variation among the sites.

\begin{center}\rule{0.5\linewidth}{0.5pt}\end{center}

Figure 3. Results of the spatially-explicit models predicting substrate
density and water content of the peat, which are defined as the mass (in
grams) of solids and water per litre of uncompacted peat. Predictions
are presented on the maps as rainbow colors and observed values at the
sampling locations are presented with dots using the same rainbow color
scale as for the model predictions. The substrate density model
(\(P^2=0.088\)) is much weaker than the water content model
(\(P^2=0.25\)).

\begin{center}\rule{0.5\linewidth}{0.5pt}\end{center}

Figure 4. Site loadings of a two-axis PCA over the Oribatid mite study
area. These axes represent the two main components of the Oribatid mite
community structure (representing approximately 25\% of the among-site
variability). Rainbow colors pixels on the surface of the study area are
the values obtained from predicted counts, whereas the background color
of the markers correspond to the observed PCA axis loadings.

\begin{center}\rule{0.5\linewidth}{0.5pt}\end{center}

\pagebreak

Table 1. Distance-weighting functions, from Dray, Legendre, and
Peres-Neto (2006), commonly used for Moran's eigenvector maps
calculation.

\begin{longtable}[]{@{}
  >{\raggedright\arraybackslash}p{(\columnwidth - 4\tabcolsep) * \real{0.1667}}
  >{\raggedright\arraybackslash}p{(\columnwidth - 4\tabcolsep) * \real{0.7436}}
  >{\raggedleft\arraybackslash}p{(\columnwidth - 4\tabcolsep) * \real{0.0897}}@{}}
\toprule\noalign{}
\begin{minipage}[b]{\linewidth}\raggedright
Name
\end{minipage} & \begin{minipage}[b]{\linewidth}\raggedright
Definition
\end{minipage} & \begin{minipage}[b]{\linewidth}\raggedleft
\end{minipage} \\
\midrule\noalign{}
\endhead
\bottomrule\noalign{}
\endlastfoot
Linear & \(a_{i,j} = 1-\frac{d_{i,j}}{d_{max}}\) & (T1~1) \\
Concave up & \(a_{i,j} = 1-\left(\frac{d_{i,j}}{d_{max}}\right)^\alpha\)
& (T1~2) \\
Concave down & \(a_{i,j} = \frac{1}{d_{i,j}^\alpha}\) & (T1~3) \\
\end{longtable}

Notes:

\begin{itemize}
\item
  \(d_{max}\): the maximum distance for two points to be considered
  neighbours, also referred to as the range parameter
\item
  \(\alpha\): a shape parameter
\end{itemize}

\pagebreak

Table 2. Distance-weighting functions usable for the generation of
predictive Moran's Eigenvector Maps, which are based on the classical
MEM framework (1--3) of variogram models (4--7).

\begin{longtable}[]{@{}
  >{\raggedright\arraybackslash}p{(\columnwidth - 4\tabcolsep) * \real{0.1076}}
  >{\raggedright\arraybackslash}p{(\columnwidth - 4\tabcolsep) * \real{0.8206}}
  >{\raggedleft\arraybackslash}p{(\columnwidth - 4\tabcolsep) * \real{0.0717}}@{}}
\toprule\noalign{}
\begin{minipage}[b]{\linewidth}\raggedright
Name
\end{minipage} & \begin{minipage}[b]{\linewidth}\raggedright
Definition
\end{minipage} & \begin{minipage}[b]{\linewidth}\raggedleft
\end{minipage} \\
\midrule\noalign{}
\endhead
\bottomrule\noalign{}
\endlastfoot
\(\mathrm{Linear^1}\) & \(w_i = \begin{cases}
          d_i < d_{max}, 1-\frac{d_i}{d_{max}} \\
          d_i \geq d_{max}, 0
         \end{cases}\) & \(\mathrm{T2 1}\) \\
\(\mathrm{Power^1}\) & \(w_i = \begin{cases}
          d_i < d_{max}, 1-\left(\frac{d_i}{d_{max}}\right)^\alpha \\
          d_i \geq d_{max}, 0
          \end{cases}\) & \(\mathrm{T2 2}\) \\
\(\mathrm{Hyperbolic^1}\) & \(w_i = \begin{cases}
          d < d_{max}, \frac{\left(1 + \frac{d_i}{d_{max}} \right)^{-\alpha} - 2^{-\alpha}}{1 - 2^{-\alpha}} \\
          d_i \geq d_{max}, 0
          \end{cases}\) & \(\mathrm{T2 3}\) \\
\(\mathrm{Spherical}\) & \(w_i = \begin{cases}
          d_i < d_{max}, 1 - 1.5\left(\frac{d_i}{d_{max}}\right) +  0.5\left(\frac{d_i}{d_{max}}\right)^3 \\
          d_i \geq d_{max}, 0
        \end{cases}\) & \(\mathrm{T2 4}\) \\
\(\mathrm{Exponential}\) & \(w_i = \mathrm{e}^{-\frac{d_i}{d_{max}}}\) &
\(\mathrm{T2 5}\) \\
\(\mathrm{Gaussian}\) &
\(w_i = \mathrm{e}^{-\left(\frac{d_i}{d_{max}}\right)^2}\) &
\(\mathrm{T2 6}\) \\
\(\mathrm{Hole effect}\) & \(w_i = \begin{cases}
          d_i = 0, 1 \\
          d_i > 0, \frac{d_{max}}{\pi d_i}\sin \frac{\pi d_i}{d_{max}}
        \end{cases}.\) & \(\mathrm{T2 7}\) \\
\end{longtable}

\textbf{Notes:}

\begin{enumerate}
\def\labelenumi{\arabic{enumi}.}
\tightlist
\item
  See Appendix~I -- Distance weighting function derived from the MEM
  framework for a through presentation of these three DWF.
\end{enumerate}

\pagebreak

Table 3. Results of the analysis of variance of the effect of the sample
size (\(\log_{10}n\)), distance weighting functions (\(DWF\)), maps
(\(Map\)), and samples (\(Sample\)) on the coefficient of prediction
(\(P^2\)) of all the models generated during the simulation study. The
analysis also included the second and third order interaction terms.

\begin{longtable}[]{@{}
  >{\raggedright\arraybackslash}p{(\columnwidth - 6\tabcolsep) * \real{0.4935}}
  >{\raggedleft\arraybackslash}p{(\columnwidth - 6\tabcolsep) * \real{0.0779}}
  >{\raggedright\arraybackslash}p{(\columnwidth - 6\tabcolsep) * \real{0.2597}}
  >{\raggedright\arraybackslash}p{(\columnwidth - 6\tabcolsep) * \real{0.1688}}@{}}
\toprule\noalign{}
\begin{minipage}[b]{\linewidth}\raggedright
\end{minipage} & \begin{minipage}[b]{\linewidth}\raggedleft
\(\nu\)
\end{minipage} & \begin{minipage}[b]{\linewidth}\raggedright
\(F_{\nu,\nu_{res}}\)
\end{minipage} & \begin{minipage}[b]{\linewidth}\raggedright
\(P\)
\end{minipage} \\
\midrule\noalign{}
\endhead
\bottomrule\noalign{}
\endlastfoot
\(\log_{10}n\) & 1 & 337800 & \(< 0.000\,1\) \\
\(DWF\) & 6 & 225.4 & \(< 0.000\,1\) \\
\(Map\) & 24 & 3782 & \(< 0.000\,1\) \\
\(Sample\) & 24 & 38.04 & \(< 0.000\,1\) \\
\(\log_{10}n \times DWF\) & 6 & 174.5 & \(< 0.000\,1\) \\
\(\log_{10}n \times Map\) & 24 & 263.4 & \(< 0.000\,1\) \\
\(\log_{10}n \times Sample\) & 24 & 34.3 & \(< 0.000\,1\) \\
\(DWF \times Map\) & 144 & 1.936 & \(< 0.000\,1\) \\
\(DWF \times Sample\) & 144 & 0.4698 & \(> 0.05\) \\
\(Map \times Sample\) & 576 & 8.796 & \(< 0.000\,1\) \\
\(\log_{10}n \times DWF \times Map\) & 144 & 1.58 & \(< 0.000\,1\) \\
\(\log_{10}n \times DWF \times Sample\) & 144 & 0.6375 & \(> 0.05\) \\
\(\log_{10}n \times Map \times Sample\) & 576 & 8.13 & \(< 0.000\,1\) \\
\(DWF \times Map \times Sample\) & 3456 & 0.3894 & \(> 0.05\) \\
\(\mathrm{Residuals}\) & 20956 & & \\
\end{longtable}

\pagebreak

Table 4. Results of the analysis of variance of the effect of the sample
size (\(\log_{10}n\)), maps (\(Map\)), and samples (\(Sample\)),
including their second and third order interaction terms, on the
coefficient of prediction (\(P^2\)) of the models with the best distance
weighting functions; the latter is the one providing the highest value
of the prediction quality metric for any map and subset combinations.

\begin{longtable}[]{@{}
  >{\raggedright\arraybackslash}p{(\columnwidth - 6\tabcolsep) * \real{0.4935}}
  >{\raggedleft\arraybackslash}p{(\columnwidth - 6\tabcolsep) * \real{0.0779}}
  >{\raggedright\arraybackslash}p{(\columnwidth - 6\tabcolsep) * \real{0.2597}}
  >{\raggedright\arraybackslash}p{(\columnwidth - 6\tabcolsep) * \real{0.1688}}@{}}
\toprule\noalign{}
\begin{minipage}[b]{\linewidth}\raggedright
\end{minipage} & \begin{minipage}[b]{\linewidth}\raggedleft
\(\nu\)
\end{minipage} & \begin{minipage}[b]{\linewidth}\raggedright
\(F_{\nu,\nu_{res}}\)
\end{minipage} & \begin{minipage}[b]{\linewidth}\raggedright
\(P\)
\end{minipage} \\
\midrule\noalign{}
\endhead
\bottomrule\noalign{}
\endlastfoot
\(\log_{10}n\) & 1 & 55980 & \(< 0.000\,1\) \\
\(Map\) & 24 & 669.7 & \(< 0.000\,1\) \\
\(Sample\) & 24 & 7.49 & \(< 0.000\,1\) \\
\(\log_{10}n \times Map\) & 24 & 39.11 & \(< 0.000\,1\) \\
\(\log_{10}n \times Sample\) & 24 & 6.794 & \(< 0.000\,1\) \\
\(Map \times Sample\) & 576 & 1.515 & \(< 0.000\,1\) \\
\(\log_{10}n \times Map \times Sample\) & 576 & 1.423 &
\(< 0.000\,1\) \\
\(\mathrm{Residuals}\) & 2500 & & \\
\end{longtable}

\end{document}
